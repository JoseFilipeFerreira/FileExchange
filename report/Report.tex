\documentclass[a4paper]{report}
\usepackage[utf8]{inputenc}
\usepackage[portuguese]{babel}
\usepackage{hyperref}
\usepackage{a4wide}
\hypersetup{pdftitle={Sistema de Troca de Ficheiros},
pdfauthor={José Ferreira, João Teixeira, Miguel Solino},
colorlinks=true,
urlcolor=blue,
linkcolor=black}
\usepackage{subcaption}
\usepackage[cache=false]{minted}
\usepackage{listings}
\usepackage{booktabs}
\usepackage{multirow}
\usepackage{appendix}
\usepackage{tikz}
\usepackage{authblk}
\usepackage{verbatim}
\usetikzlibrary{positioning,automata,decorations.markings}
\AfterEndEnvironment{figure}{\noindent\ignorespaces}

\begin{document}

\title{Troca de Ficheiros\\ 
\large Grupo Nº 34}
\author{José Ferreira (A83683) \and João Teixeira (A85504) \and Miguel Solino
(A86435)}
\date{\today}

\begin{center}
    \begin{minipage}{0.75\linewidth}
        \centering
        \includegraphics[width=0.4\textwidth]{eng.jpeg}\par\vspace{1cm}
        \vspace{1.5cm}
        \href{https://www.uminho.pt/PT}
        {\color{black}{\scshape\LARGE Universidade do Minho}} \par
        \vspace{1cm}
        \href{https://www.di.uminho.pt/}
        {\color{black}{\scshape\Large Departamento de Informática}} \par
        \vspace{1.5cm}
        \maketitle
    \end{minipage}
\end{center}

\tableofcontents

\pagebreak
\chapter{Introdução}
O objetivo deste projeto é desenvolver um sistema de partilha de ficheiros de 
música que seja capaz de gerir os ficheiros assim como os respetivos
meta-dados.\\
A arquitetura a ser utilizada deve seguir a lógica \textit{Cliente, Servidor}
com recurso a \textit{Sockets TCP} e \textit{Threads}, aplicando, assim, os
conhecimentos sobre Sistemas Distribuídos leccionados durante as aulas da UC.\\
Como objetivo, é necessário implementar um sistema que permita a utilização 
concorrente de um servidor por vários clientes, assegurando a consistência
de toda a informação introduzida no sistema para qualquer cliente que esteja
a fazer uso da mesma.\\
Ao longo deste relatório vamos descrever as nossas abordagens a estes problemas.

\chapter{Arquitetura e Solução do Projeto}
\section{Servidor}
No modulo Servidor são tratadas todas as conexões e pedidos dos clientes,
garantindo a capacidade de suportar vários clientes em simultâneo.\\
Para além disso, é tratada a gestão da informação relativa a logins e a
músicas armazenadas de forma a que seja possível um acesso concorrente entre as
diversas \textit{Threads} lançadas.\\
A recessão, envio e armazenamento dos ficheiros de música e também tratado neste
modulo, tendo em vista uma boa gestão de recursos, tanto a nível de memoria
como a nível do \textit{CPU}.\\
De forma a conseguir um servidor capaz de lidar com vários clientes em
simultâneo, este é dividido em varias \textit{Threads}, sendo que é criada uma
nova para tratar cada cliente que se tente conectar, uma para receber e
responder aos pedidos vindos do cliente, uma para enviar notificações sobre
novas músicas adicionadas, uma para gerir os \textit{Downloads} requisitados, e
uma ultima para receber os ficheiros de música enviados pelo respetivo cliente.

\section{Cliente}
No modulo Cliente é efetuada a ponte entre o utilizador e o servidor, sendo 
capaz de efetuar pedidos ao servidor e apresentar ao utilizador a resposta aos
pedidos efetuados. Este está preparado para complementar o servidor no envio e
recessão de ficheiros de música.

\section{Comunicação Cliente-Servidor}
\subsection{Pedidos e respostas aos pedidos}
Para efetuar a comunicação entre o cliente e o servidor fixamos
um formato de texto, inspirado no formato \textit{JSON}, que no caso
de pedidos ao servidor tem o formato \ref{inputformat}, onde o 
\textit{request\_type} corresponde ao tipo de pedido que o cliente quer
efetuar.\\
Por exemplo, para adicionar uma música iria ser \textit{add\_music}.\\

\begin{figure}[H]
    \begin{verbatim}
        {type='request_type', content=['obj1','obj2',...]}
    \end{verbatim}
    \caption{Formato dos pedidos}
    \label{inputformat}
\end{figure}
As respostas a pedidos segue o formato \ref{outputformat}.\\
Neste, o \textit{status} pode ser \textit{ok} ou \textit{failed} e o
\textit{res} é a resposta ao pedido efetuado ou a razão pela qual não foi
possível  efectuá-lo.

\begin{figure}[H]
    \begin{verbatim}
        {status='status', result='res'}
    \end{verbatim}
    \caption{Formato das repostas}
    \label{outputformat}
\end{figure}
Para comunicar com o cliente, linhas de texto com este formato são transmitidos
através de uma socket criada para o efeito.

\subsection{Uploads e Downloads}
Para efetuar a transferência de ficheiros de e para o servidor, é criada uma
\textit{ServerSocket} por cada cliente, numa porta aleatória, e posteriormente
e enviada ao servidor a informação para ele se conectar a essa \textit{Socket}.
Nesta \textit{socket} apenas são enviados binários correspondentes ao ficheiro
de música, em partes de um tamanho máximo definido. Embora sejam possíveis
downloads e uploads concorrentes entre clientes, um cliente apenas pode efetuar
um download e um upload em simultâneo.

\subsection{Notificação de novas músicas}
Quando é adicionada uma nova música ao sistema, esta é adicionada a uma
\textit{queue} de notificações que será utilizada por cada \textit{thread} de
notificações, enviando assim uma notificação a todos os clientes que estejam
ligados ao servidor no momento.

\chapter{Conclusão}
Para concluir, conseguimos cumprir todos os requisitos propostos, criando no processo um
sistema de troca de ficheiros de música capaz de utilizar uma arquitetura 
\textit{Cliente, Servidor} que suporta vários clientes em simultâneo.
Como trabalho futuro, gostaríamos de permitir a cada cliente ter mais do que um
download e upload a correr em simultâneo.

\end{document}
